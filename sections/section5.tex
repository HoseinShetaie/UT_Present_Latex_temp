\section{Conclusion}

\begin{frame}{Challenges}
\begin{itemize}
    \item  \textbf{Electric field provided to the model in forward problem}  
    \quad In the paper, the electric field $E(x,t)$ was directly supplied to the PINN during forward training. this may limit the model's ability to infer $E$ self-consistently from the physics.

    \vspace{0.3cm}
    \item  \textbf{Integration in the Poisson equation}  
    \quad The Poisson term $\displaystyle \frac{\partial E}{\partial x} = \int f\,dv - 1$ requires numerical integration.  
    The choice of integration method (e.g., trapezoidal rule) affects stability and accuracy.

    \vspace{0.3cm}
    \item  \textbf{Unknown weighting of loss components}  
    \quad The relative importance of each loss term ($L_{\text{eq}}, L_{\text{IC}}, L_{\text{BC}}$)  
    is not clearly defined-improper weighting can hinder convergence or bias the network.

    \vspace{0.3cm}
    \item  \textbf{Sampling and resampling strategy}  
    \quad The performance strongly depends on how training points are chosen in $(t,x,v)$.  
    Optimal sampling frequency and resampling intervals remain open research questions.
\end{itemize}
\pause
\begin{block}{Observation}
Balancing physical accuracy, numerical stability, and training efficiency  
is still the main challenge for PINN-Vlasov frameworks.
\end{block}
\end{frame}

\begin{frame}{Conclusion}
\begin{itemize}
    \item \textbf{Physics-Informed Neural Networks (PINNs)} bridge the gap between deep learning and physical modeling.
    \item By embedding \textbf{governing PDEs} into the loss function, they ensure physical consistency.
    \item Demonstrated capability for solving both \textbf{forward and inverse problems} with high interpretability.
\end{itemize}
\end{frame}

\begin{frame}{Future Work}
    \begin{itemize}
    \item  \textbf{Improving computational efficiency} — faster training for complex PDEs.
    \item  \textbf{Adaptive sampling} — dynamic selection of collocation points.
    \item  \textbf{Multi-physics extension} — include magnetic fields, collisions, higher dimensions (2D/3D).

\end{itemize}
\vspace{0.5cm}
\pause
\centering
\textit{Future PINNs = Physics + Data + Efficiency → Scientific AI.}
\end{frame}

\backmatter
